\documentclass[a4paper,1em]{article}
\usepackage{siunitx}
\usepackage{mathtools}
\usepackage{gensymb}
\usepackage{amsmath,amssymb,cases}
\usepackage{graphicx}
\usepackage{minted}
\usepackage{caption}
\usepackage{listings}
\usepackage{hyperref}
\usepackage{cleveref}
\usepackage{setspace}
\usepackage{parskip}
\usepackage{cite}
\usepackage{tikz}
\usepackage{svg}
\usepackage{pxfonts}
\usepackage{geometry}
\usepackage{mdframed}
\usetikzlibrary{decorations.pathmorphing,patterns,arrows.meta}

\geometry{
    a4paper,
    margin=1.2in,
}
\title{
    Recipes
}
\date{\today}
\author{
    Damien \textsc{Bezborodov}
}

\newenvironment{code}{\captionsetup{type=listing}}{}

\hypersetup{
    colorlinks=true,
    linkcolor=black,
    citecolor=black,
    filecolor=black,
    urlcolor=black,
    pdftitle={Recipes},
}
\lstset{basicstyle=\ttfamily}
\onehalfspacing


\begin{document}

\maketitle

\tableofcontents

\newpage
\section{Irish Stew}

Lamb shoulder, onion, potatoes, carrots, parsley, garlic (optional), Worcestershire sauce (optional), pepper, flour (optional.) Water or beef stock.

Beef stock is optional and spring water is sufficient as the lamb provides enough flavour as-is.

(Served with cabbage, onion and bacon with white wine vinegar. Sides of cheese, bread, butter, and walnuts.)

Method:


\begin{enumerate}
    \item Prepare lamb.
    \begin{enumerate}
        \item Dice.
        \item Coat with flour, salt, and pepper.
        \item Caramelise in small batches and set aside.
    \end{enumerate}
    \item Roast garlic in the lamb fat, and set aside.
    \item Dice largely the carrots, potatoes, and onions.
    \item Optionally brown the vegetables in the lamb fat also.
    \item Add meat, garlic, and vegetables to the pot with water (or beef stock.)
    \item Flavour with Worcestershire sauce.
    \item Bring to a boil.
    \item Simmer for 2–3 hours.
    \item Towards the end, add parsley leaves with stems removed, saving some for garnish.
\end{enumerate}

\end{document}
