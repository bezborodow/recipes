\documentclass[a4paper,1em]{article}
\usepackage{siunitx}
\usepackage{mathtools}
\usepackage{gensymb}
\usepackage{amsmath,amssymb,cases}
\usepackage{graphicx}
\usepackage{minted}
\usepackage{caption}
\usepackage{listings}
\usepackage{hyperref}
\usepackage{cleveref}
\usepackage{setspace}
\usepackage{parskip}
\usepackage{cite}
\usepackage{tikz}
\usepackage{svg}
\usepackage{pxfonts}
\usepackage{geometry}
\usepackage{mdframed}
\usetikzlibrary{decorations.pathmorphing,patterns,arrows.meta}

\geometry{
    a4paper,
    margin=1.2in,
}
\title{
    Recipes
}
\date{\today}
\author{
    Damien \textsc{Bezborodov}
}

\newenvironment{code}{\captionsetup{type=listing}}{}

\hypersetup{
    colorlinks=true,
    linkcolor=black,
    citecolor=black,
    filecolor=black,
    urlcolor=black,
    pdftitle={Recipes},
}
\lstset{basicstyle=\ttfamily}
\onehalfspacing


\begin{document}

\maketitle

\tableofcontents

\newpage
\section{Irish Stew}

Lamb shoulder, onion, potatoes, carrots, parsley, garlic (optional),
Worcestershire sauce (optional), pepper, flour (optional.) Spring water or beef stock.

Beef stock is optional and spring water is sufficient as the lamb provides enough flavour as-is.

(Served with cabbage and bacon with white wine vinegar. Sides of cheese, bread, butter, and walnuts.)

Method:

\begin{enumerate}
    \item Prepare lamb.
    \begin{enumerate}
        \item Dice into large cubes (3/4 of an inch.)
        \item Fat will render down and add flavour and nutrients to the stew. However,
            excessive fat should be removed. Gristel should be removed.
        \item Coat with flour, salt, and pepper.
        \item Add butter to the pan.
        \item Caramelise in small batches and set aside.
            Do not introduce excess flour into the pan when caramelising,
            otherwise burnt granules of flour will end up in the stew. Place
            cubes of lamb individually into the pan without excess flour.  Do
            not rush. Remove individually also with care when putting aside.
            Pan should be hot enough to caramelise the meat, but not too hot,
            otherwise the meat and flour coating will burn.
        \item Roast very gently the garlic in the lamb fat, and set aside.
        \item Déglacer the pan with Worcestershire sauce followed by water.
    \end{enumerate}
    \item Dice largely the carrots, potatoes, and onions.
    \item Optionally brown the vegetables in the lamb fat also.
    \item Add meat, garlic, and vegetables to the pot with water (or beef stock.)
    \item Bring to a boil.
    \item Simmer for 2–3 hours.
    \item Towards the end, add parsley leaves with stems removed, saving some for garnish.
\end{enumerate}

\end{document}
